% % Lines to compile only this capter
% \documentclass[11pt, twoside, a4paper, openright]{report}
% \usepackage[utf8]{inputenc}
% \DeclareUnicodeCharacter{223C}{~}

%Bibliography style
% \usepackage[square, numbers]{natbib}
% \usepackage[round]{natbib}
% \usepackage{biblatex}
% \bibliographystyle{unsrtnat}
% \bibliographystyle{unsrt}
% \bibliographystyle{plain}
% \bibliographystyle{aa}
% \usepackage[backend=bibtex,style=authoryear,natbib=true]{biblatex} 
\usepackage[
backend=biber,
style=authoryear,
citestyle=authoryear,
url=false
]{biblatex}
\addbibresource{../source/library.bib}

\usepackage[T1]{fontenc}
\usepackage[french]{babel}
\usepackage{csquotes}  % used for citations (recommended when using biblatex)
%\usepackage{helvet}
%\renewcommand{\familydefault}{\sfdefault}
\usepackage{mathptmx}
\usepackage{amssymb}
\usepackage{geometry} 
\usepackage{xcolor}
\usepackage[absolute,overlay]{textpos}
\usepackage{graphicx}
\usepackage{lipsum}
\usepackage[explicit]{titlesec}
\usepackage{lmodern}
\usepackage{color}
\usepackage{array}
\usepackage{mathtools}
\usepackage{caption}
\usepackage{multicol}
\usepackage{booktabs}
\usepackage{enumitem}
\usepackage{hyperref}
\usepackage{afterpage}
\usepackage{emptypage}
\usepackage{setspace}
\usepackage{pgffor}
    \setlength{\columnseprule}{0pt}
    \setlength\columnsep{10pt}
\usepackage[francais,nohints]{minitoc}
    \setcounter{minitocdepth}{3}
 
 %https://la-bibliotex.fr/2019/02/03/ecrire-les-nombres-et-les-unites-avec-latex/   
\usepackage{siunitx}
% \sisetup{
%     detect-all,
%      output-decimal-marker={,},
%      group-minimum-digits = 3,
%      group-separator={~},
%      number-unit-separator={~},
%      inter-unit-product={~},
%      list-separator = {, },
%      list-final-separator = { et },
%      range-phrase = --,
%      separate-uncertainty = true,
%      multi-part-units = single,
%      list-units = single,
%      range-units = single
%     }
\usepackage{physics}
\usepackage{isotope}

\usepackage[perpage]{footmisc} % to reset the counter of footnote each page

    
\usepackage{fancyhdr}			% Entête et pieds de page. Doit être placé APRES geometry
\pagestyle{fancy}		% Indique que le style de la page sera justement fancy
%\lfoot[\thepage]{} 		% gauche du pied de page
%\cfoot{} 			% milieu du pied de page
%\rfoot[]{\thepage} 
\fancyfoot{} % vide le pied~de~page
\fancyfoot[LE,RO]{\thepage}
\fancyfoot[LO,CE]{}% droite du pied de page
\fancyhead{}	
\fancyhead[LE]{\leftmark}	
\fancyhead[RO]{\rightmark}

\fancypagestyle{plain}{%
\fancyhf{} % vide l’en-tête et le pied~de~page.
\fancyfoot[LE,RO]{\thepage} % numéro de la page en cours en gras% et centré en pied~de~page.
\renewcommand{\headrulewidth}{0pt}
\renewcommand{\footrulewidth}{0pt}}



% Premiere page des chapitres
\newlength\chapnumb
\setlength\chapnumb{3cm}
 
\titleformat{\chapter}[block] {
  \normalfont}{}{0pt} { %police
    \parbox[b]{\chapnumb}{
      \fontsize{120}{110}\selectfont\thechapter} %taille du chiffre
      \parbox[b]{\dimexpr\textwidth-\chapnumb\relax}{
        \raggedleft 
        \hfill{\bfseries\Huge#1}\\ %taille du titre
        \rule{\dimexpr\textwidth-\chapnumb\relax}{0.4pt} %ligne de separation
  }
}
 
 %premiere page chapitre non numerote (remerciement, table des matieres ...)
 
\titleformat{name=\chapter,numberless}[block]
{\normalfont}{}{0pt}
{   
    \parbox[b]{\dimexpr\textwidth}{%   
    \hfill{\bfseries\Huge#1}\\
  \rule{\dimexpr\textwidth}{0.4pt}}}
    
 %   \titleformat{name=\chapter,numberless}[block]
%{\normalfont}{}{0pt}
%{\parbox[b]{\chapnumb}{%
%   \mbox{}}%
%  \parbox[b]{\dimexpr\textwidth-\chapnumb\relax}{%
%    \raggedleft%
%    \hfill{\bfseries\Huge#1}\\
%    \rule{\dimexpr\textwidth-\chapnumb\relax}{0.4pt}}}


%%%    SIunitx
\sisetup{locale = FR,
  % inter-unit-product=\ensuremath{\cdot},
  inter-unit-product=\ensuremath{\,},
  per-mode=reciprocal,
  separate-uncertainty = true,
  detect-all
}
\DeclareSIUnit{\Mpc}{Mpc}
\DeclareSIUnit{\kpc}{kpc}
\DeclareSIUnit{\Gpc}{Gpc}
\DeclareSIUnit{\h}{\textit{h}~}
\DeclareSIUnit{\perh}{\textit{h}^{-1}\,}

%%% Geometry
\geometry{
left=20mm,
top=30mm,
right=20mm,
bottom=30mm
}

%%% Color
\definecolor{bordeau}{rgb}{0.3515625,0,0.234375}

%%% Commands
\newcommand{\Nmocks}{\num{30}}
\newcommand{\hMpc}{h^{-1}\,\mathrm{Mpc}}
\newcommand{\hGpc}{h^{-1}\,\mathrm{Gpc}}
\newcommand{\kms}{\mathrm{km\,s^{-1}}}

\newcommand{\lya}{Ly$\alpha$}
\newcommand{\lyb}{Ly$\beta$}
\newcommand{\lyalya}{Ly$\alpha$(Ly$\alpha$)}
\newcommand{\lyalyb}{Ly$\alpha$(Ly$\beta$)}

\newcommand{\lrf}{\lambda_{\rm RF}}
\newcommand{\kpar}{k_{\parallel}}
\newcommand{\apar}{\alpha_{\parallel}}
\newcommand{\rpar}{r_{\parallel}}
\newcommand{\aperp}{\alpha_{\perp}}
\newcommand{\rperp}{r_{\perp}}
\newcommand{\kperp}{k_{\perp}}

\newcommand{\blya}{b_{\rm Ly\alpha}}
\newcommand{\betalya}{\beta_{\rm Ly\alpha}}
\newcommand{\blyb}{b_{\rm Ly\alpha}}
\newcommand{\betalyb}{\beta_{\rm Ly\beta}}
\newcommand{\dlya}{d_{\rm Ly\alpha}}
\newcommand{\bhcd}{b_{\rm HCD}}
\newcommand{\betahcd}{\beta_{\rm HCD}}
\newcommand{\Fhcd}{F_{\rm HCD}}
\newcommand{\Lhcd}{L_{\rm HCD}}

\newcommand{\imin}{i_{\rm min}}
\newcommand{\imax}{i_{\rm max}}
\newcommand{\jmin}{j_{\rm min}}
\newcommand{\jmax}{j_{\rm max}}

\newcommand{\xioned}{\xi_{\rm 1d}}
\newcommand{\DHub}{D_{H}}
\newcommand{\DM}{D_{M}}

\newcommand{\omegam}{\Omega_M}
\newcommand{\omegac}{\Omega_C}
\newcommand{\omegab}{\Omega_B}
\newcommand{\omegan}{\Omega_\nu}
\newcommand{\omegal}{\Omega_\Lambda}
\newcommand{\omegak}{\Omega_k}
\newcommand{\orad}{\Omega_R}
\newcommand{\ogam}{\Omega_\gamma}
\newcommand{\lcdm}{$\Lambda$CDM}

\newcommand{\picca}{\texttt{picca}}

%%% Rem's command
\newcommand\blankpage{%
    \null
    \thispagestyle{empty}%
    \addtocounter{page}{-1}%
    \newpage}
  
% Command to set up a particular alignment for a cell in tabular :
% \myalign{c}{foo} for instance
\newcommand*{\myalign}[2]{\multicolumn{1}{#1}{#2}}
 
\renewcommand{\thesection}{\arabic{section}}

% Romain
\newcommand{\cRM}[1]{\MakeUppercase{\romannumeral #1}}	% Capital
\newcommand{\cRm}[1]{\textsc{\romannumeral #1}}	% Petit majuscule
\newcommand{\crm}[1]{\romannumeral #1}
% Siècle %
\newcommand{\siecle}[1]{\cRm{#1}\textsuperscript{e}~siècle}



% Thesis title
\newcommand{\PhDTitle}{Les forêts \lya{} du relevé eBOSS : comprendre les fonctions de corrélation et les systématiques} 

% Name
\newcommand{\PhDname}{Thomas Etourneau} 

% Change this variable if you add or remove chapters
\newcommand*{\NumOfChapters}{6}

% Change this variable if you add or remove appendices
\newcommand*{\NumOfAppendices}{2}

% PDF metadata
\hypersetup{
	pdfauthor={\PhDname},
	pdfsubject={Manuscrit de thèse de doctorat},
	pdftitle={\PhDTitle}
}


% \begin{document}




\chapter{Conclusion}
\thispagestyle{fancy}

Dans ce manuscrit, j'ai présenté le résultat de mes recherches effectuées au sein du groupe de cosmologie du CEA Saclay, dans le cadre de la préparation de ma thèse de doctorat.
Durant ces trois ans, j'ai développé des simulations qui imitent les données Lyman alpha (\lya{}) acquises par eBOSS et bientôt par DESI.
Ces simulations produisent un relevé de quasars, dont les forêts \lya{} présentes dans les spectres synthétiques possèdent les bonnes fonctions de corrélation.
Pour ce faire, elles utilisent les champs aléatoires gaussiens. D'abord, nous générons un champ aléatoire qui représente le champ de densité. Nous donnons à ce champ aléatoire la bonne fonction de corrélation à l'aide des transformations de Fourier.
Puis, grâce à une transformation log-normale, nous tirons un relevé de quasars. Ensuite, à partir de chaque quasars, nous reconstruisons la densité le long de la ligne de visée en interpolant ce champ. Enfin, nous obtenons la fraction de flux transmise en appliquant l'approximation de Gunn-Peterson (FGPA) au champ interpolé le long de la ligne de visée.
Ceci nous permet d'obtenir la forêt d'absorption \lya{} pour chaque quasar de notre relevé.
Aussi, nous calculons la prédiction des fonctions de corrélation que nous devons obtenir avec ces simulations. Ceci nous permet de vérifier leur construction en comparant la fonction de corrélation prédite à la fonction de corrélation mesurée dans nos simulations.
Enfin, l'analyse des fonctions de corrélation issues des mocks (chapitre~\ref{chap:mock_ana}) montre que les mocks sont en très bon accord avec la prédiction, et qu'ils sont très bien décrits par les modèles utilisés pour ajuster les données.


Ces simulations ont été utilisées dans l'analyse qui mesure l'échelle des oscillations acoustiques de baryon (BAO) à partir des données finales d'eBOSS DR16~\autocite{DuMasdesBourboux2020} afin de vérifier que cette analyse produisait des mesures non biaisées de l'échelle BAO. Elles seront aussi utilisées par la collaboration DESI afin de vérifier les analyses \lya{}.
% Par ailleurs, ces simulations ont été utilisées pour produire une carte tomographique du champ de matière à grande échelle à l'aide des données d'eBOSS.
Par ailleurs, ces simulations ont été utilisées par \textcite{Ravoux2020} pour produire une carte tomographique du champ de matière à grande échelle à partir des données d'eBOSS. Elles ont permis de valider l'algorithme utilisé pour reconstruire ce champ de matière.

Nos simulations, avec celles présentées dans \textcite{Farr2019}, sont à la pointe des programmes qui simulent la distribution d'absorption \lya{} à grande échelle. Ces simulations sont un atout majeur et indispensable aux analyses de mesure de l'échelle BAO et de mesure des distorsions dans l'espace des redshifts (RSD) avec le \lya{}. Les simulations que nous présentons dans ce manuscrit ont l'avantage de posséder une fonction de corrélation prédictible, ce qui est très utile pour les analyses RSD.
Cette prédiction nous permet d'ajuster efficacement les valeurs du biais $b_{\mathrm{Ly}\alpha}$, du paramètre RSD $\beta_{\mathrm{Ly}\alpha}$, de la transmission moyenne $\overline F$ et du spectre de puissance $P^{\mathrm{1D}}(k)$ à chaque redshift.
De plus, ces simulations bénéficient aussi de distributions de quasars et HCD (High Column Density) très proches de celles observées dans les données, ce qui permet d'obtenir des fonctions de corrélation croisées entre les différents éléments des simulations très correctes. Afin d'améliorer encore ces simulations, il conviendrait d'ajouter les absorptions causées par les HCD et les métaux directement lors de la construction des forêts \lya{}, avant l'ajout du continuum de chaque quasar.



\paragraph{}
Grâce à l'étude approfondie de nos simulations (chapitres~\ref{chap:mock_ana} et~\ref{chap:data_ana}), nous montrons que l'analyse des données finales d'eBOSS DR16, présentée dans \textcite{DuMasdesBourboux2020}, produit une mesure des paramètres BAO $\apar{}$ et $\aperp{}$ non biaisée à \SI{1}{\percent} près.
Nous montrons aussi que nous comprenons la forme de la fonction de corrélation \lya{}$\times$\lya{}, c'est à dire la mesure du biais $b_{\mathrm{Ly}\alpha}$ et du paramètre RSD $\beta_{\mathrm{Ly}\alpha}$ du \lya{}, à environ \SI{20}{\percent} près.
Pour affiner cette compréhension, nous devons comprendre pourquoi les distorsions induites sur la fonction de corrélation \lya{}$\times$\lya{} par l'ajustement du continuum des quasars ne sont pas complètement prises en compte par la matrice de distorsion.
Ceci doit passer par une comparaison approfondie des simulations brutes (les raw mocks) et des simulations avec continuum. Afin de simplifier les comparaisons, il est prévu de produire des mocks avec continuum sans ajouter le bruit de mesure et sans inclure la diversité des quasars.

Lors de l'étude des simulations avec HCD (chapitre~\ref{chap:data_ana}), nous montrons que la modélisation des HCD n'est pas parfaite, et qu'elle mène à des mesures du biais $b_{\mathrm{Ly}\alpha}$ et du paramètre RSD $\beta_{\mathrm{Ly}\alpha}$ du \lya{} biaisée.
Premièrement, la valeur du paramètre $L_{\textsc{HCD}}$ dans l'analyse des données d'eBOSS (présentée dans \textcite{DuMasdesBourboux2020}), fixée à \SI{10}{\perh\Mpc}, est mal choisie. Nous suggérons d'utiliser $L_{\textsc{HCD}} = \SI{2.8}{\perh\Mpc}$ si le modèle de Rogers est utilisé.
Deuxièmement, nous étudions une modélisation alternative des HCD, que nous appelons le modèle C-G (chapitre~\ref{chap:data_ana}). Ce modèle produit des mesures compatibles sur les mocks. Cependant, des différences restent à comprendre lorsque nous ajustons les données, notamment lorsque nous ajustons les données avec HCD masqués ou sans masquer les HCD. Suite à ces observations, nous privilégions la piste d'un effet inconnu, présent dans les données (et non dans les mocks) et étant à l'origine de ces différences.
Tant que notre compréhension des HCD et de leur modélisation n'aura pas évoluée, nous pensons qu'il est préférable d'utiliser le modèle C-G, sans masquer les HCD. Ceci pour deux raisons : le fait de ne pas masquer les HCD permet de disposer d'une plus grande statistique pour mesurer la position du pic BAO et les paramètre \lya{}. De plus, les paramètres \lya{} sont moins corrélés dans un tel cas. La seconde raison est que le fait de masquer les HCD peut potentiellement introduire des systématiques dans la fonction de corrélation : les pixels d'absorptions identifiés comme HCD et masqués dans l'analyse ne correspondent pas à une partie aléatoire des données, mais à des régions très denses du ciel. Les masquer peut introduire des systématiques dans l'analyse.


\paragraph{}
Avec l'arrivée du relevé DESI, une grande quantité de données, et donc une grande puissance statistique, sera mise à la disposition de la communauté scientifique. Cette puissance statistique va permettre d'effectuer des mesures BAO avec le \lya{} à différents redshifts, et peut-être de mener une première analyse RSD avec le \lya{}. Dans ce but, un effort doit être fait pour comprendre les problèmes liés à la matrice de distorsion et à la modélisation des HCD. Un effort doit aussi être fait pour comprendre les différences qu'il y a entre les mesures faites sur l'auto-corrélation \lya{}$\times$\lya{} et les mesures faites sur la corrélation croisée \lya{}$\times$QSO (chapitre~\ref{chap:mock_ana}).
Nous montrons dans ce manuscrit que les simulations que nous avons développées permettent d'étudier ces effets et ainsi de comprendre les potentielles systématiques qui affectent ces mesures. Ces simulations sont donc un atout majeur pour préparer les analyses \lya{} du relevé DESI.


% \printbibliography
% \end{document}
