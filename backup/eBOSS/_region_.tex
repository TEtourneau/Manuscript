\message{ !name(doc.tex)}\documentclass[11pt, twoside, a4paper, openright]{report}

\usepackage[utf8x]{inputenc}
\usepackage[T1]{fontenc}
\usepackage[francais]{babel}
\usepackage{amsmath} % or simply amstext
\usepackage{geometry}
\usepackage{SIunits}
\usepackage{graphicx}
\geometry{left=3cm, right=3cm}

% \usepackage{booktabs}
% \usepackage{lineno}
% \usepackage{graphicx}
% \usepackage[usenames,dvipsnames,svgnames,table]{xcolor} %% To color in violet the text

% \usepackage[absolute]{textpos} % to place elements in the page
% \usepackage{multicol} % To write summary in two columns-mode
% \usepackage{calc} % To calculate textwidth



\begin{document}

\message{ !name(doc.tex) !offset(143) }



\subsection{Résultats}

\#prov Ca sera plus simple à remplir une fois que le papier de cosmo eBOSS sera sorti.

\section{DESI}

Le \emph{Dark Energy Spectroscopic Instrument} (DESI) est un projet américain de mesure d'énergie noire de génération 4. Il a vu sa première lumière en octobre 2019 et devrait commencer la prise de données en juillet 2020. (\#prov Commissionning octobre 2019 - fev 2020 puis 3 mois de SV pour tester la TS, qualité des spectres pour la détermination du redshift (temps d'exposition), puis dernier mois de SV oú on fait 1~\% du survey avec la config choisie (TS entre autre). A la fin de ce mois là, soit on garde et on continue le survey, soit on retouche la TS par exemple, et on part pour le survey (dans ce cas le 1~\% est perdu).)

Comme eBOSS, DESI étudie les BAO et la croissance des structures à l'aide d'un très grand relevé de galaxies et de quasars. A l'issue des 5 ans d'observation prévus, DESI aura mesuré plus de 30 millions de spectres, distribués sur un relevé de plus de 14~000~$deg²$. \\
Pour atteindre ses objectifs, DESI utilise le télescope Mayall, mesurant 4~m de diamètre et situé au Kitt Peak en Arizona. Le champ de vue du télescope est le même que celui de SDSS : 3$\degres$ de diamètre sur le ciel. L'instrument inclut aussi un système de fibre optique, au nombre de 5~000, mais celles ci sont placées au plan focal à l'aide de robots qui ajustent la position de chaque fibre avant chaque exposition. 10 spectrographes reçoivent ces fibres, chacun comportant 3 caméras et couvrant les longueurs d'onde de 3~600 à 9~800~$\angstrom$. \\
% L'instrument inclut 5~000 fibres, dont le placement au plan focal est robotisé. Les fibres sont envoyés vers 10 spectrographes, chacun comportant 3 caméras et couvrant les longueurs d'onde de 3~600 à 9~800~$\angstrom$. \\
‌‌DESI utilise les 4 mêmes traceurs qu'eBOSS : les LRG jusqu'à $z=1,0$, les ELG jusqu'à $z=1,7$, ainsi que les quasars en tant que traceurs directs de la matière et les quasars Ly$\alpha$ sur la gamme $2,1 < z < 3,5$. En plus de ces traceurs, DESI obervera des galaxies brillantes (BG : \emph{Bright Galaxies}) pendant le dark time (\#prov). Le relevé de ces galaxies contiendra 10 millions d'objets, avec un redshift moyen $z=0,2$.

A la fin des 5 ans d'observations, DESI fournira plus de 30 points de mesure de distance, chacun avec une précision plus petite que le pourcent, et couvrant la gamme $0 < z < 3.5$. La figure~\ref{DesiVsEboss} illustre la différence entre BOSS et DESI pour la mesure de distance de Hubble.\\
De plus, DESI donnera une mesure de la somme des masses des neutrinos, avec une incentitude de 0,020~eV. Cette précision est su
\message{ !name(doc.tex) !offset(149) }

\end{document}
