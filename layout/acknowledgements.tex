% created on 2019-12-13
% @author : bmazoyer
\chapter*{Remerciements}
\thispagestyle{plain}


I would like first to thank my examinations board who kindly accepted to be part of this board, listen to and evaluate the completion of this 3 years research work.
I would like to thank especially Delphine Hardin and Matthew Pieri for reading this whole manuscript with caution and giving precious advice. I know it is a lot of work, in particular in the middle of August when everyone enjoy the summer. So thank you very much.
I would like to give warm thanks to Andreu Font-Ribera for being part of my examinations board, but also and particularly for the help he gives to the community. You really care about people, especially about students, who may feel lost in these big collaborations. Your advice and support are very precious, and it is important to have people like you in these collaborations. So thank you very much for the time and energy you spend for us.
I would also like to thank the eBOSS and DESI collaborations, and in particular the Lyman-$\alpha$ working group of these collaborations, for their non-competitive and friendly atmosphere. I will never forget the valuable meetings we had and all these nights we spent in bars playing pool.

\paragraph{}
J'aimerais maintenant remercier Jean-Marc Le Goff, mon directeur de thèse, pour ces trois années passées à ses côtés. Jean-Marc, tu es quelqu'un de très humain, et j'ai tout de suite apprécié cette qualité chez toi. Tu as su me faire confiance très vite, et cela m'a permis de gagner en maturité (j'en avais grandement besoin !). Tu es aussi quelqu'un de très patient, tu prends la peine d'expliquer et réexpliquer, toujours calmement, ce qui est d'autant plus appréciable. Jamais je ne t'ai vu t'énerver. Finalement, ces trois années passées en ta compagnie ont été très enrichissantes, autant scientifiquement que humainement. J'ai beaucoup changé durant ces trois années, ma vision de la vie a évolué, et c'est, en parti, grâce à tes enseignements. Merci d'avoir été aussi dévoué, et d'avoir investi autant de temps et d'énergie dans ma formation (et dans la relecture de ma thèse !).

J'aimerais aussi remercier le groupe cosmologie du CEA Saclay pour son accueil chaleureux. Je me suis très vite senti à l'aise avec vous. J'ai particulièrement apprécié les discussions, scientifiques ou non, que nous pouvions avoir le midi ou durant les réunions de groupe. Un grand merci à vous, pour tous ces moments passés ensemble, et pour toutes les fois où vous vous êtes rendus disponibles pour me donner un coup de main, pour m'expliquer ce point de science, ou cette curiosité que j'avais.

Enfin, je remercie bien évidemment tous les doctorants et post-doc du groupe cosmologie, avec qui, il faut le dire, je me suis bien marrer. Je remercie Manu, mon co-bureau, pour tous les débats passionnant et incongrus que tu as su lancer sans aucunes prémices. Je remercie aussi le fameux 41b, ainsi que Solène et Richard (dans leur luxueux bureau), pour tous ces moments de rigolade, mais aussi, tous ces moments de mal aise tellement amusants. Merci aussi pour tous ces \emph{Friday lunch} mails, écrits avec beaucoup d'humour et surtout beaucoup d'imagination. J'espère que cette tradition perdurera.

\paragraph{}
J'en viens à remercier du fond du cœur mes proches, pour simplement avoir été là lorsqu'il le fallait.
J'aimerais remercier premièrement ma famille, et en particulier mes parents et mon frère, pour m'avoir supporté, puis soutenu, durant toutes mes études. Maman, Papa, merci pour les sacrifices que vous avez pu faire, afin de rendre possible ces études. Vous nous avez toujours soutenus, Luc et moi, toujours cru en nous, faisant de notre bonheur et notre épanouissement une priorité. Je me suis toujours senti accompagné, jamais senti seul, et c'est grâce à vous que j'en suis là aujourd'hui. Merci.

J'en viens maintenant à mes amis. Vous ne faites pas parti de ma famille, mais pour moi c'est tout comme. J'aimerais remercier dans un premier temps la bande de potes que l'on a formé à Périgueux, depuis les années lycée. Ce groupe m'est précieux, car c'est avec vous que je me suis construit dès la fin de mon adolescence. Je me remémore avec nostalgie nos nombreuses soirées passées à faire les cons, nos nombreuses vacances en camping ... Je sais qu'il y en aura d'autres, et il me tarde. Merci pour tous ces moments, merci d'avoir été là.
J'aimerais aussi remercier M. Rémy Thoër, Rems12, et M. Guillaume Ventura, Smeag, pour former ce trio magique qui a su accompagner nos années universitaires. Rémy, on t'a rencontré lors de ce fameux week-end d'intégration. Celui-ci nous a soudé et on ne s'est plus jamais lâché. Je vous témoigne ici toute la gratitude pour ce support mutuel que nous nous sommes apportés, et pour cette bonne humeur qui a fait que ces années se sont si bien déroulées.
Finalement, j'aimerais remercier la bande de potes qui s'est construite à Orsay. Qui aurait cru que cette école d'été à Fréjus serait à l'origine de tant de rencontres, de tant d'amour ? 
