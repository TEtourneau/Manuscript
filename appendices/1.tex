% % created on 2019-12-13
% % @author : bmazoyer
% % Lines to compile only this capter
% \documentclass[11pt, twoside, a4paper, openright]{report}
% \usepackage[utf8]{inputenc}
% \DeclareUnicodeCharacter{223C}{~}

%Bibliography style
% \usepackage[square, numbers]{natbib}
% \usepackage[round]{natbib}
% \usepackage{biblatex}
% \bibliographystyle{unsrtnat}
% \bibliographystyle{unsrt}
% \bibliographystyle{plain}
% \bibliographystyle{aa}
% \usepackage[backend=bibtex,style=authoryear,natbib=true]{biblatex} 
\usepackage[
backend=biber,
style=authoryear,
citestyle=authoryear,
url=false
]{biblatex}
\addbibresource{../source/library.bib}

\usepackage[T1]{fontenc}
\usepackage[french]{babel}
\usepackage{csquotes}  % used for citations (recommended when using biblatex)
%\usepackage{helvet}
%\renewcommand{\familydefault}{\sfdefault}
\usepackage{mathptmx}
\usepackage{amssymb}
\usepackage{geometry} 
\usepackage{xcolor}
\usepackage[absolute,overlay]{textpos}
\usepackage{graphicx}
\usepackage{lipsum}
\usepackage[explicit]{titlesec}
\usepackage{lmodern}
\usepackage{color}
\usepackage{array}
\usepackage{mathtools}
\usepackage{caption}
\usepackage{multicol}
\usepackage{booktabs}
\usepackage{enumitem}
\usepackage{hyperref}
\usepackage{afterpage}
\usepackage{emptypage}
\usepackage{setspace}
\usepackage{pgffor}
    \setlength{\columnseprule}{0pt}
    \setlength\columnsep{10pt}
\usepackage[francais,nohints]{minitoc}
    \setcounter{minitocdepth}{3}
 
 %https://la-bibliotex.fr/2019/02/03/ecrire-les-nombres-et-les-unites-avec-latex/   
\usepackage{siunitx}
% \sisetup{
%     detect-all,
%      output-decimal-marker={,},
%      group-minimum-digits = 3,
%      group-separator={~},
%      number-unit-separator={~},
%      inter-unit-product={~},
%      list-separator = {, },
%      list-final-separator = { et },
%      range-phrase = --,
%      separate-uncertainty = true,
%      multi-part-units = single,
%      list-units = single,
%      range-units = single
%     }
\usepackage{physics}
\usepackage{isotope}

\usepackage[perpage]{footmisc} % to reset the counter of footnote each page

    
\usepackage{fancyhdr}			% Entête et pieds de page. Doit être placé APRES geometry
\pagestyle{fancy}		% Indique que le style de la page sera justement fancy
%\lfoot[\thepage]{} 		% gauche du pied de page
%\cfoot{} 			% milieu du pied de page
%\rfoot[]{\thepage} 
\fancyfoot{} % vide le pied~de~page
\fancyfoot[LE,RO]{\thepage}
\fancyfoot[LO,CE]{}% droite du pied de page
\fancyhead{}	
\fancyhead[LE]{\leftmark}	
\fancyhead[RO]{\rightmark}

\fancypagestyle{plain}{%
\fancyhf{} % vide l’en-tête et le pied~de~page.
\fancyfoot[LE,RO]{\thepage} % numéro de la page en cours en gras% et centré en pied~de~page.
\renewcommand{\headrulewidth}{0pt}
\renewcommand{\footrulewidth}{0pt}}



% Premiere page des chapitres
\newlength\chapnumb
\setlength\chapnumb{3cm}
 
\titleformat{\chapter}[block] {
  \normalfont}{}{0pt} { %police
    \parbox[b]{\chapnumb}{
      \fontsize{120}{110}\selectfont\thechapter} %taille du chiffre
      \parbox[b]{\dimexpr\textwidth-\chapnumb\relax}{
        \raggedleft 
        \hfill{\bfseries\Huge#1}\\ %taille du titre
        \rule{\dimexpr\textwidth-\chapnumb\relax}{0.4pt} %ligne de separation
  }
}
 
 %premiere page chapitre non numerote (remerciement, table des matieres ...)
 
\titleformat{name=\chapter,numberless}[block]
{\normalfont}{}{0pt}
{   
    \parbox[b]{\dimexpr\textwidth}{%   
    \hfill{\bfseries\Huge#1}\\
  \rule{\dimexpr\textwidth}{0.4pt}}}
    
 %   \titleformat{name=\chapter,numberless}[block]
%{\normalfont}{}{0pt}
%{\parbox[b]{\chapnumb}{%
%   \mbox{}}%
%  \parbox[b]{\dimexpr\textwidth-\chapnumb\relax}{%
%    \raggedleft%
%    \hfill{\bfseries\Huge#1}\\
%    \rule{\dimexpr\textwidth-\chapnumb\relax}{0.4pt}}}


%%%    SIunitx
\sisetup{locale = FR,
  % inter-unit-product=\ensuremath{\cdot},
  inter-unit-product=\ensuremath{\,},
  per-mode=reciprocal,
  separate-uncertainty = true,
  detect-all
}
\DeclareSIUnit{\Mpc}{Mpc}
\DeclareSIUnit{\kpc}{kpc}
\DeclareSIUnit{\Gpc}{Gpc}
\DeclareSIUnit{\h}{\textit{h}~}
\DeclareSIUnit{\perh}{\textit{h}^{-1}\,}

%%% Geometry
\geometry{
left=20mm,
top=30mm,
right=20mm,
bottom=30mm
}

%%% Color
\definecolor{bordeau}{rgb}{0.3515625,0,0.234375}

%%% Commands
\newcommand{\Nmocks}{\num{30}}
\newcommand{\hMpc}{h^{-1}\,\mathrm{Mpc}}
\newcommand{\hGpc}{h^{-1}\,\mathrm{Gpc}}
\newcommand{\kms}{\mathrm{km\,s^{-1}}}

\newcommand{\lya}{Ly$\alpha$}
\newcommand{\lyb}{Ly$\beta$}
\newcommand{\lyalya}{Ly$\alpha$(Ly$\alpha$)}
\newcommand{\lyalyb}{Ly$\alpha$(Ly$\beta$)}

\newcommand{\lrf}{\lambda_{\rm RF}}
\newcommand{\kpar}{k_{\parallel}}
\newcommand{\apar}{\alpha_{\parallel}}
\newcommand{\rpar}{r_{\parallel}}
\newcommand{\aperp}{\alpha_{\perp}}
\newcommand{\rperp}{r_{\perp}}
\newcommand{\kperp}{k_{\perp}}

\newcommand{\blya}{b_{\rm Ly\alpha}}
\newcommand{\betalya}{\beta_{\rm Ly\alpha}}
\newcommand{\blyb}{b_{\rm Ly\alpha}}
\newcommand{\betalyb}{\beta_{\rm Ly\beta}}
\newcommand{\dlya}{d_{\rm Ly\alpha}}
\newcommand{\bhcd}{b_{\rm HCD}}
\newcommand{\betahcd}{\beta_{\rm HCD}}
\newcommand{\Fhcd}{F_{\rm HCD}}
\newcommand{\Lhcd}{L_{\rm HCD}}

\newcommand{\imin}{i_{\rm min}}
\newcommand{\imax}{i_{\rm max}}
\newcommand{\jmin}{j_{\rm min}}
\newcommand{\jmax}{j_{\rm max}}

\newcommand{\xioned}{\xi_{\rm 1d}}
\newcommand{\DHub}{D_{H}}
\newcommand{\DM}{D_{M}}

\newcommand{\omegam}{\Omega_M}
\newcommand{\omegac}{\Omega_C}
\newcommand{\omegab}{\Omega_B}
\newcommand{\omegan}{\Omega_\nu}
\newcommand{\omegal}{\Omega_\Lambda}
\newcommand{\omegak}{\Omega_k}
\newcommand{\orad}{\Omega_R}
\newcommand{\ogam}{\Omega_\gamma}
\newcommand{\lcdm}{$\Lambda$CDM}

\newcommand{\picca}{\texttt{picca}}

%%% Rem's command
\newcommand\blankpage{%
    \null
    \thispagestyle{empty}%
    \addtocounter{page}{-1}%
    \newpage}
  
% Command to set up a particular alignment for a cell in tabular :
% \myalign{c}{foo} for instance
\newcommand*{\myalign}[2]{\multicolumn{1}{#1}{#2}}
 
\renewcommand{\thesection}{\arabic{section}}

% Romain
\newcommand{\cRM}[1]{\MakeUppercase{\romannumeral #1}}	% Capital
\newcommand{\cRm}[1]{\textsc{\romannumeral #1}}	% Petit majuscule
\newcommand{\crm}[1]{\romannumeral #1}
% Siècle %
\newcommand{\siecle}[1]{\cRm{#1}\textsuperscript{e}~siècle}



% Thesis title
\newcommand{\PhDTitle}{Les forêts \lya{} du relevé eBOSS : comprendre les fonctions de corrélation et les systématiques} 

% Name
\newcommand{\PhDname}{Thomas Etourneau} 

% Change this variable if you add or remove chapters
\newcommand*{\NumOfChapters}{6}

% Change this variable if you add or remove appendices
\newcommand*{\NumOfAppendices}{2}

% PDF metadata
\hypersetup{
	pdfauthor={\PhDname},
	pdfsubject={Manuscrit de thèse de doctorat},
	pdftitle={\PhDTitle}
}



% \begin{document}

\chapter{Fluctuating Gunn Peterson Approximation}
\label{app:fgpa}

L'approximation FGPA (Fluctuating Gunn Peterson Approximation), présentée dans la section~\ref{subsubsec:lya_field}, relie la profondeur optique $\tau$ au contraste de densité $\delta$ :
\begin{equation}
  \tau = a \exp(b \delta) \; ,
\end{equation}
où $a$ et $b$ sont deux paramètres.
Dans cette approximation, l'élargissement thermique des raies d'absorption est négligé.

\paragraph{}
Afin de relier la profondeur optique $\tau$ au contraste de densité $\delta$, nous débutons avec l'équation qui donne l'équilibre de photo-ionisation de l'hydrogène neutre ($\mathrm{H}_{\mathrm{I}}$) :
\begin{equation}
  \gamma + \mathrm{H}_{\mathrm{I}} \leftrightarrow \mathrm{e}^{-} + \mathrm{p}^{+} \; ,
\end{equation}
qui se traduit comme
\begin{equation}
  n_{\gamma} n_{\mathrm{H}_{\mathrm{I}}} \langle \sigma_{\mathrm{ioni}} c \rangle
  = n_{\mathrm{e}} n_{\mathrm{p}} \langle \sigma_{\mathrm{rec}} v \rangle_{T} \; ,
\end{equation}
où $n_{\gamma}$, $n_{\mathrm{H}_{\mathrm{I}}}$, $n_{\mathrm{e}}$ et $n_{\mathrm{p}}$ donnent respectivement les densités de photons, d'hydrogènes neutres, d'électrons et de protons, 
et $ \sigma_{\mathrm{ioni}}$ et $\sigma_{\mathrm{rec}}$ donnent respectivement les sections efficaces de l'ionisation d'un atome d'hydrogène neutre et de la recombinaison d'un électron et d'un proton.
% En sachant que les densités d'électron et de protons sont égales à la densité de baryons, nous pouvons exprimer la densité d'hydrogène neutre comme
En négligeant l'hélium et parce que l'hydrogène est quasiment totalement ionisé (entre 1 atome sur \num{10000} et 1 atome sur \num{1000} demeure neutre), nous pouvons approximer les densités d'électron et de protons par la densité de baryons. La densité d'hydrogène neutre s'exprime alors comme
\begin{equation}
  n_{\mathrm{H}_{\mathrm{I}}}
  = n_{b}^2 \frac{\langle \sigma_{\mathrm{rec}} v \rangle_{T}}{n_{\gamma} \langle \sigma_{\mathrm{ioni}} c \rangle } \; .
\end{equation}
% De plus, le terme $\langle \sigma_{\mathrm{rec}} v \rangle_{T}$ est proportionnel à $T^{-\num{0.7}}$, nous avons donc la relation de proportionnalité
De plus, nous avons $\langle \sigma_{\mathrm{rec}} v \rangle_{T} \propto T^{-\num{0.7}}$, nous avons donc la relation de proportionnalité
\begin{equation}
  n_{\mathrm{H}_{\mathrm{I}}} \propto \frac{(1+z)^6 \Omega_b^2 (1+\delta)^2 T^{-\num{0.7}}}{n_{\gamma} \langle \sigma_{\mathrm{ioni}} c \rangle} \; ,
\end{equation}
où nous avons explicité la densité de baryon $n_{b}^2 = (1+z)^6 \Omega_b^2 (1+\delta)^2$ en fonction du contraste de densité $\delta$.
Nous pouvons relier de la même manière la température $T$ au contraste de densité $\delta$ comme
\begin{equation}
  T = \overline T(z)(1+\delta)^{\gamma(z) -1 } \; ,
\end{equation}
où $\gamma (z \sim 3) \sim \num{1.6}$ \autocite{Hui1996}.
Nous pouvons ainsi relier la profondeur optique $\tau$ au contraste de densité comme
\begin{equation}
  \label{eq:tau_vs_delta}
  \tau(z) \propto  \Omega_b^2 \frac{(1+z)^6 \overline T(z)^{-\num{0.7}}}{H(z)J_{\gamma}}
  \frac{(1+\delta)^{\beta}}{1+\eta} \; ,
\end{equation}
où $\eta$ est le gradient de vitesse $\eta = v'_{p}(z) / H(z)$, $H(z)J_{\gamma}$ correspond au flux ionisant de photons, et $\beta = 2 - \num{0.7}(\gamma(z) -1) \sim \num{1.6}$ à $z \sim 3$.
% De l'équation précédente, nous déduisons que, pour des faibles gradients de vitesse et pour $z \sim 3$
% \begin{equation}
% \tau(z) \propto \exp(\num{1.6} \times \delta) (1 - \eta) \; .
% \end{equation}
% Cette équation correspond à l'approximation FGPA (Fluctuating Gunn Peterson Approximation).

\paragraph{}
Lorsque nous nous intéressons uniquement au contraste de densité $\delta$, l'équation~\ref{eq:tau_vs_delta} et l'approximation log-normale \autocite{coles_lognormal_1991} nous donnent 
\begin{equation}
\tau(z) \propto \exp(\beta \delta) \; ,
\end{equation}
avec $\beta \sim \num{1.6}$ à $z \sim 3$.
Cette équation correspond à l'approximation FGPA (Fluctuating Gunn Peterson Approximation).
Lorsque nous prenons en compte le gradient de vitesse $\eta$, nous obtenons une version légèrement modifiée de l'approximation FGPA
\begin{equation}
\tau(z) \propto \exp(\beta \delta - \eta) \; .
\end{equation}
Cette équation est l'approximation que nous utilisons pour construire les mocks présentés dans ce manuscrit.


% \printbibliography
% \end{document}
