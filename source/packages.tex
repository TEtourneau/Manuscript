\usepackage[utf8]{inputenc}

\usepackage[hyperref=true,
            url=false,
            isbn=false,
            backref=true,
            backend=biber,
            style=authoryear,
            citestyle=authoryear,
            maxcitenames=2,
            maxbibnames=3]{biblatex}
\usepackage{csquotes}  % used for citations (recommended when using biblatex)
\addbibresource{../source/library.bib}
\usepackage[T1]{fontenc}
\usepackage[cyr]{aeguill}  % pour les guillemets et accents francais
\usepackage{xspace}  % pour babel francais
\usepackage[francais]{babel}


\usepackage{mathptmx}
\usepackage{amssymb}
\usepackage{geometry} 
\usepackage{color}
\usepackage{xcolor}
% \usepackage[absolute,overlay]{textpos}
\usepackage{graphicx}
\usepackage{lipsum}
\usepackage[explicit]{titlesec}
\usepackage{lmodern}
\usepackage{array}
\usepackage{mathtools}
\usepackage{caption}
\usepackage{booktabs}
\usepackage{enumitem}
\usepackage[hyperfootnotes=false,colorlinks=true]{hyperref}
\usepackage[perpage]{footmisc}  % to reset the counter of footnote each page
\usepackage{afterpage}
\usepackage{emptypage}
\usepackage{setspace}
\usepackage{pgffor}
    \setlength{\columnseprule}{0pt}
    \setlength\columnsep{10pt}
\usepackage[francais,nohints]{minitoc}
    \setcounter{minitocdepth}{3}
 
 %https://la-bibliotex.fr/2019/02/03/ecrire-les-nombres-et-les-unites-avec-latex/   
\usepackage{siunitx}
% \sisetup{
%     detect-all,
%      output-decimal-marker={,},
%      group-minimum-digits = 3,
%      group-separator={~},
%      number-unit-separator={~},
%      inter-unit-product={~},
%      list-separator = {, },
%      list-final-separator = { et },
%      range-phrase = --,
%      separate-uncertainty = true,
%      multi-part-units = single,
%      list-units = single,
%      range-units = single
%     }
% \usepackage{physics}
% \usepackage{isotope}

%%% Package du template Paris Saclay
\usepackage{amsmath}
\usepackage{amsfonts}
\definecolor{Prune}{RGB}{99,0,60}
\usepackage{mdframed}
\usepackage{multirow} %% Pour mettre un texte sur plusieurs rangées
\usepackage{multicol} %% Pour mettre un texte sur plusieurs colonnes
\usepackage{scrextend} %Forcer la 4eme  de couverture en page pair
\usepackage{tikz}
\usepackage[absolute]{textpos} 
\usepackage{colortbl}
%\RequirePackage{geometry}% That nicely create a one-page template
%\geometry{textheight=100ex,textwidth=40em,top=30pt,headheight=30pt,headsep=30pt,inner=80pt}
% \usepackage{geometry}


    
\usepackage{fancyhdr}			% Entête et pieds de page. Doit être placé APRES geometry
\pagestyle{fancy}		% Indique que le style de la page sera justement fancy
%\lfoot[\thepage]{} 		% gauche du pied de page
%\cfoot{} 			% milieu du pied de page
%\rfoot[]{\thepage} 
\fancyfoot{} % vide le pied~de~page
\fancyfoot[LE,RO]{\thepage}
\fancyfoot[LO,CE]{}% droite du pied de page
\fancyhead{}	
\fancyhead[LE]{\leftmark}	
\fancyhead[RO]{\rightmark}

\fancypagestyle{plain}{%
\fancyhf{} % vide l’en-tête et le pied~de~page.
\fancyfoot[LE,RO]{\thepage} % numéro de la page en cours en gras% et centré en pied~de~page.
\renewcommand{\headrulewidth}{0pt}
\renewcommand{\footrulewidth}{0pt}}

\renewcommand{\contentsname}{Sommaire}  % Sommaire au lieu de Table des matieres

% Premiere page des chapitres
\newlength\chapnumb
\setlength\chapnumb{3cm}
 
\titleformat{\chapter}[block] {
  \normalfont}{}{0pt} { %police
    \parbox[b]{\chapnumb}{
      \fontsize{120}{110}\selectfont\thechapter} %taille du chiffre
      \parbox[b]{\dimexpr\textwidth-\chapnumb\relax}{
        \raggedleft 
        \hfill{\bfseries\Huge#1}\\ %taille du titre
        \rule{\dimexpr\textwidth-\chapnumb\relax}{0.4pt} %ligne de separation
  }
}
 
 %premiere page chapitre non numerote (remerciement, table des matieres ...)
 
\titleformat{name=\chapter,numberless}[block]
{\normalfont}{}{0pt}
{   
    \parbox[b]{\dimexpr\textwidth}{%   
    \hfill{\bfseries\Huge#1}\\
  \rule{\dimexpr\textwidth}{0.4pt}}}
    
 %   \titleformat{name=\chapter,numberless}[block]
%{\normalfont}{}{0pt}
%{\parbox[b]{\chapnumb}{%
%   \mbox{}}%
%  \parbox[b]{\dimexpr\textwidth-\chapnumb\relax}{%
%    \raggedleft%
%    \hfill{\bfseries\Huge#1}\\
%    \rule{\dimexpr\textwidth-\chapnumb\relax}{0.4pt}}}


%%%    SIunitx
\sisetup{locale = FR,
  % inter-unit-product=\ensuremath{\cdot},
  inter-unit-product=\ensuremath{\,},
  per-mode=reciprocal,
  separate-uncertainty = true,
  detect-all
}
\DeclareSIUnit{\Mpc}{Mpc}
\DeclareSIUnit{\kpc}{kpc}
\DeclareSIUnit{\Gpc}{Gpc}
\DeclareSIUnit{\h}{\textit{h}~}
\DeclareSIUnit{\perh}{\textit{h}^{-1}\,}

%%% Geometry
\geometry{
left=20mm,
top=30mm,
right=20mm,
bottom=30mm
}

%%% Color
\definecolor{bordeau}{rgb}{0.3515625,0,0.234375}

%%% Commands
\newcommand{\Nmocks}{\num{30}}
\newcommand{\hMpc}{h^{-1}\,\mathrm{Mpc}}
\newcommand{\hGpc}{h^{-1}\,\mathrm{Gpc}}
\newcommand{\kms}{\mathrm{km\,s^{-1}}}

\newcommand{\lya}{Ly$\alpha$}
\newcommand{\lyb}{Ly$\beta$}
\newcommand{\lyalya}{Ly$\alpha$(Ly$\alpha$)}
\newcommand{\lyalyb}{Ly$\alpha$(Ly$\beta$)}

\newcommand{\lrf}{\lambda_{\rm RF}}
\newcommand{\kpar}{k_{\parallel}}
\newcommand{\apar}{\alpha_{\parallel}}
\newcommand{\rpar}{r_{\parallel}}
\newcommand{\aperp}{\alpha_{\perp}}
\newcommand{\rperp}{r_{\perp}}
\newcommand{\kperp}{k_{\perp}}

\newcommand{\blya}{b_{\rm Ly\alpha}}
\newcommand{\betalya}{\beta_{\rm Ly\alpha}}
\newcommand{\blyb}{b_{\rm Ly\alpha}}
\newcommand{\betalyb}{\beta_{\rm Ly\beta}}
\newcommand{\dlya}{d_{\rm Ly\alpha}}
\newcommand{\bhcd}{b_{\rm HCD}}
\newcommand{\betahcd}{\beta_{\rm HCD}}
\newcommand{\Fhcd}{F_{\rm HCD}}
\newcommand{\Lhcd}{L_{\rm HCD}}

\newcommand{\imin}{i_{\rm min}}
\newcommand{\imax}{i_{\rm max}}
\newcommand{\jmin}{j_{\rm min}}
\newcommand{\jmax}{j_{\rm max}}

\newcommand{\xioned}{\xi_{\rm 1d}}
\newcommand{\DHub}{D_{H}}
\newcommand{\DM}{D_{M}}

\newcommand{\omegam}{\Omega_M}
\newcommand{\omegac}{\Omega_C}
\newcommand{\omegab}{\Omega_B}
\newcommand{\omegan}{\Omega_\nu}
\newcommand{\omegal}{\Omega_\Lambda}
\newcommand{\omegak}{\Omega_k}
\newcommand{\orad}{\Omega_R}
\newcommand{\ogam}{\Omega_\gamma}
\newcommand{\lcdm}{$\Lambda$CDM}

\newcommand{\picca}{\texttt{picca}}

%%% Rem's command
\newcommand\blankpage{%
    \null
    \thispagestyle{empty}%
    \addtocounter{page}{-1}%
    \newpage}
  
% Command to set up a particular alignment for a cell in tabular :
% \myalign{c}{foo} for instance
\newcommand*{\myalign}[2]{\multicolumn{1}{#1}{#2}}
 
% \renewcommand{\thesection}{\arabic{section}}

% Romain
\newcommand{\cRM}[1]{\MakeUppercase{\romannumeral #1}}	% Capital
\newcommand{\cRm}[1]{\textsc{\romannumeral #1}}	% Petit majuscule
\newcommand{\crm}[1]{\romannumeral #1}
% Siècle %
\newcommand{\siecle}[1]{\cRm{#1}\textsuperscript{e}~siècle}



% Thesis title
\newcommand{\PhDTitle}{Les forêts Lyman-$\alpha$ du relevé eBOSS : comprendre les fonctions de corrélation et les systématiques} 

% Name
\newcommand{\PhDname}{Thomas Etourneau} 

% Change this variable if you add or remove chapters
\newcommand*{\NumOfChapters}{6}

% Change this variable if you add or remove appendices
\newcommand*{\NumOfAppendices}{2}

% PDF metadata
\hypersetup{
	pdfauthor={\PhDname},
	pdfsubject={Manuscrit de thèse de doctorat},
	pdftitle={\PhDTitle}
}
